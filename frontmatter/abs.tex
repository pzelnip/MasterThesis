\newpage
\TOCadd{Abstract}

\noindent \textbf{Supervisory Committee}
\tpbreak
\panel

\begin{center}
\textbf{ABSTRACT}
\end{center}

This thesis examines the pedagogical value of a particular visual programming environment (VPE) called the Gem Cutter which is based upon the functional programming paradigm.  The contribution of this thesis is two-fold: it provides a qualitative evaluation via the Cognitive Dimensions Framework developed by Green to explore the usefulness of the Gem Cutter environment from a pedagogical viewpoint, and secondly provides a framework called the Word Game Framework designed in the Gem Cutter which can be used to create exercises for sudents learning to program. The Word Game Framework allows students to create interactive turn-based ``word-games'' and provides an engaging environment for students to explore interesting and useful functional programming concepts such as recursion, higher order functions, type inference, and list processing.