\newpage
\TOCadd{Abstract}

\noindent \textbf{Supervisory Committee}
\tpbreak
\panel

\begin{center}
\textbf{ABSTRACT}
\end{center}

This thesis examines the pedagogical value of a particular visual programming environment (VPE) called the Gem Cutter which is based upon the functional programming paradigm.  The contribution of this thesis is two-fold: it gives a set of ``cookbook-style'' learning exercises which instructors can either use as-is or modify to suit their own needs, and secondly provides an evaluation of the usefulness of the Gem Cutter environment from a pedagogical viewpoint.  The learning exercises revolve around the design and implementation of a framework that allows students to create interactive turn-based ``word-games'' called the Word Game Framework. This framework provides an engaging environment for students to explore interesting and useful functional programming concepts such as recursion, higher order functions, type inference, and list processing.  The evaluation of the visual programming environment is done using the Cognitive Dimensions Framework.