
\startappendix{Word Game Framework Implementation Details}
\label{chapter:wgfHelpers}

\section{Helper Routines}
\label{wordGameHelper}
In creating the \(game()\) routine of the Word Game Framework, we also devised a set of routines to handle the various I/O tasks.  Each is outlined in detail in the subsequent sections.

\subsection{session}

\(session()\) is a routine which encapsulates the basic gameplay session.  That is, it tests if the game has ended using the supplied gameOverTest predicate, and if so returns the final string displayed to the player and the final game state as a tuple.  More formally session is defined as:

\code{
\begin{tabbing}
session\=::\\
\>stateChangingFunction :: a -> String -> (String, a)\\
\>stateToStringFunction :: a -> String\\
\>initialState :: a\\
\>userPrompt :: String\\
\>gameOverTest :: (String, a) -> Boolean\\
\>returns (String, a)\\
\end{tabbing}
}

Each parameter corresponds to one of the arguments passed to the main \(game()\) routine.  A screenshot of the implementation of \(session()\) is shown in \fref{sessionImplementation}.  Note that this function also has a local function called \(endGameTest()\) defined within it.  This function is simply the logical or of the two possible ways of a game ending: either by the gameOverTest predicate returning true, or by the player entering the ``quit'' command.

\insertFigure{4.5}{sessionImplementation}{Implementation of the session() gem}

\subsection{processLine}

\(processLine()\) is a routine which encapsulates a single ``move'' of the game.  That is, it represents one single application of the state transition function, and returns a tuple consisting of a message to display to the player and the new game state.  More formally, processLine is defined as:

\code{
\begin{tabbing}
processLine\=::\\
\>stateChangingFunction :: a -> String -> (String, a)\\
\>stateToStringFunction :: a -> String\\
\>previousState :: (String, a)\\
\>userPrompt :: String\\
\>returns (String, a)\\
\end{tabbing}
}

Where \code{previousState} is a tuple consisting of a String\footnote{Which is ignored, and only appears to be type compatible with the return type of \code{session}} and the state which is transitioned from, and all other parameters correspond to those passed to the main \(game()\) routine.  A screenshot of the implementation of \(processLine()\) is shown in \fref{processLineImplementation}.  Note that this function also has a local function called \(printMsg()\) defined within it which is simply for the side effect of printing the resulting string message to STDOUT.

\insertFigure{5}{processLineImplementation}{Implementation of the processLine() gem}

\subsection{answer}

\(answer()\) is a routine which applies the state transition function to the current state and current input from the player.  Additionally, answer determines if the user has prematurely quit the game by entering the special ``quit'' command.\footnote{The ``quit'' command is the only specific hard-coded command in the Word Game Framework, all others are passed to the state transition function.}  Like \(processLine()\), it returns a tuple consisting of a message to display to the player and the new game state.  More formally, answer is defined as:

\code{
\begin{tabbing}
answer\=::\\
\>stateChangingFunction :: a -> String -> (String, a)\\
\>stateToStringFunction :: a -> String\\
\>previousState :: a\\
\>userResponse :: String\\
\>returns (String, a)\\
\end{tabbing}
}

Where \code{previousState} is the state which is transitioned from, \code{userResponse} is the response read from the player, and all other parameters correspond to those passed to the main \(game()\) routine.  A screenshot of the implementation of \(answer()\) is shown in \fref{answerImplementation}.  The \code{joinStrings} code gem consists of the single line of CAL code:

\begin{verbatim}
"\n\n" ++ s ++ "\n\n" ++ stateString ++ "\n"
\end{verbatim}

which simply serves the purpose of concatenating the two output strings together with a few newlines between them.

\insertFigure{5.5}{answerImplementation}{Implementation of the answer() gem}

\subsection{goodbyeMessage}

\(goodbyeMessage()\) is a routine which simply returns the string to be displayed to the player when the game is over.  It is used to determine if and when the player prematurely ended the game by entering the ``quit'' command.  More formally, goodbyeMessage is defined as:

\code{
\begin{tabbing}
goodbyeMessage\=::\\
\>returns String\\
\end{tabbing}
}

A screenshot of the implementation of \(goodbyeMessage()\) is shown in \fref{goodbyeMessageImplementation}.

\insertFigure{3}{goodbyeMessageImplementation}{Implementation of the goodbyeMessage() gem}

\subsection{readLine}

\(readLine()\) prints a supplied prompt message to STDOUT, and then reads a line of text (using newline as a delimiter) from STDIN and returns it as a String.  More formally, readLine is defined as:

\code{
\begin{tabbing}
readLine\=::\\
\>prompt :: String\\
\>returns String\\
\end{tabbing}
}

A screenshot of the implementation of \(readLine()\) is shown in \fref{readLineImplementation}, which makes use of the Cal.Console.IO routines for reading from STDIN and printing to STDOUT.

\insertFigure{5}{readLineImplementation}{Implementation of the readLine() gem}

\subsection{quitting}

\(quitting\) is a predicate function which returns true if the supplied string is equal to ``quit'', which is the special command a player can issue to terminate a game prematurely.  More formally, quitting is defined as:

\code{
\begin{tabbing}
quitting\=::\\
\>s :: String\\
\>returns Boolean\\
\end{tabbing}
}

A screenshot of the implementation of \(quitting()\) is shown in \fref{quittingImplementation}.

\insertFigure{4}{quittingImplementation}{Implementation of the quitting() gem}
