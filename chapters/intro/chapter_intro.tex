\startfirstchapter{Introduction}
\label{chapter:introduction}

\begin{flushright}
\textit{The language used is no more than a vehicle whereby the main objective is to be achieved. It is hard for students to make this separation. While they grapple with the idiosyncrasies of whatever language they are being taught, it is very difficult to think about higher level abstract concepts.}
\\
Tony Jenkins \cite{Jenkins02} \\
\end{flushright}

Learning computer programming is hard.  So hard that oftentimes Computer Science students new to computer programming struggle
very early on in their studies, become discouraged, and move on to other disciplines before reaching the true ``science'' of Computer Science \cite{Beaubouef05}.  Most programming environments used in academic institutions are based upon ``real-world'' software development tools, designed for software development experts who are already well-versed and trained in the fundamental ideas of computer programming.  New students thus face an uphill battle: they both have to learn and understand the abstract principles, theories, and practices that instructors try to convey in early Computer Science courses, while at the same time needing to learn obscure syntactic details that are largely independent and irrelevant with respect to the learning objectives of the course.

Additionally, independent of the language, the tools and environments used play a role as well.  In a traditional compiler-based textual programming environment when students make a mistake in their program that mistake is not discovered until they finally try to compile their code.  And when they try to compile their code they are oftentimes met with cryptic error messages which discourage them at a very early stage \cite{Jenkins02,Hundhausen09,Kelleher05}.

The implication of these difficulties are revealed in the declining enrollment numbers we have seen in recent years in Computer Science \cite{Manaris07,Vesgo07,Ward08,Bayliss09}.  As such, there is considerable effort being undertaken to try and alleviate some of the common hurdles that students encounter in early CS1/CS2 style courses so as to avoid the high attrition rates seen in these courses.

One approach has been to try and remove (or diminish) the impact from issues of syntax.  In this vein, many visual programming environments (VPE's) such as Alice\cite{alice} and Scratch\cite{scratch} have been produced as part of an effort to try to help alleviate some of the common syntactic errors that can cause difficulties for students new to programming.  The hope is that this will thereby allow the key concepts instructors are trying to convey to be more readily apparent.  However most of the previously designed VPE's (and all of the VPE's that are in widespread use) are based upon the imperative and/or object-orientated programming paradigms; very few are based upon the functional programming paradigm.  This is peculiar, as the importance of the functional paradigm is increasing as time progresses, and it has been hypothesized that one of the major difficulties that students face in early CS1-style classes is learning the subtleties of the object-orientated paradigm \cite{Manaris07,Mahmoud04,Benander04}. Furthermore, there have been efforts such as the ``Teach Scheme, Reach Java'' initiative \cite{Bloch08,teachScheme,Felleisen04} that indicate that perhaps the jump from a functional programming background to a object-orientated one is less of an intellectual ``leap'' than the other way around \cite{Huch05}.

Additionally it has been suggested that one possible way to combat the declining enrollment in Computer Science is to make Computer Science courses more ``fun'' and engaging \cite{Carter06}.  One possible way for this to be done is to incorporate computer games into  curricula, and this has been done in various circumstances as a way of encouraging student involvement, motivation, and enjoyment \cite{Barnes08,Gooch08,Curtis05,Overmars05,Sweedyk05}.

As a result of these observations, this thesis has been written to explore one VPE based upon the functional paradigm, the Gem Cutter, from a pedagogical perspective by evaluating the environment as well as outline a framework written in the VPE for creating simple, turn-based games.  There are two main contributions in this thesis: an evaluation of a fully fledged visual programming environment for the purpose of aiding educators in determining if the Gem Cutter environment fits their needs, as well as some ready-made ``cookbook-style'' exercises that instructors may use in their classrooms and learning environments that make use of the game framework written in Gem Cutter.

\section{Thesis Overview}

This thesis is divided into six chapters:  

%\vspace{5mm}

%\begin{hangparas}{0.5in}{1}
\textbf{Chapter 1 Introduction} motivates the problem of learning to program, and gives an overview of the thesis.
%\end{hangparas}

%\vspace{5mm}

%\begin{hangparas}{0.5in}{1}
\textbf{Chapter 2 Background Research} explores and summarizes much of the related work to this thesis.  In particular, past and current methods of instruction are summarized, as well the examination and evaluation of some newer approaches such as the incorporation of games into computer science courses as well as the use of visual programming environments.  Lastly an outline of the overview of the Cognitive Dimensions Framework by Green is given, as well as a summary of how the framework has been applied to other programming environments.
%\end{hangparas}

\textbf{Chapter 3 The Quark Framework} gives an overview of the Quark Framework, and the two components that underly it -- the CAL language, and the Gem Cutter visual programming environment.  A general discussion of some of the strengths and weaknesses of the environment is also provided.

%\vspace{5mm}
	
%\begin{hangparas}{0.5in}{1}
\textbf{Chapter 4 The Word Game Framework} outlines the Word Game Framework developed using the Gem Cutter for allowing the creation of game-related assignments.  An overview of the framework is provided, along with a discussion of the implementation of it.  Some potential uses of the framework are explored, as well as some full, concrete, exercises that make use of the framework.
%\end{hangparas}

%\vspace{5mm}

%\begin{hangparas}{0.5in}{1}
\textbf{Chapter 5 Evaluation of the Gem Cutter Environment} evaluates the Gem Cutter environment by use of the Cognitive Dimensions Framework developed by Green.
%\end{hangparas}

%\vspace{5mm}	

%\begin{hangparas}{0.5in}{1}
\textbf{Chapter 6 Conclusion} contains a summary of the thesis and proposes some related future work.
%\end{hangparas}

\begin{comment}
\section{Challenges Presented}

What criteria needs to be met for the thesis to have met its goal?  Ie - the learning environment must have the following characteristics:

\begin{itemize}
	\item Easy to learn and use
	\item Allows new programmers to grasp fundamental programming concepts
	\item etc etc etc
\end{itemize}

\section{Approach}

\begin{itemize}
	\item Evaluation of GC via CG framework
	\item Development of Word game framework and exercises
\end{itemize}

\end{comment}

\begin{comment}
Why is this thesis useful?

Programming is hard, students new to programming often struggle to cope with the complexities of software development.

Declining enrollment, syntax errors discourage students early and thus they leave computer science before they've
gotten anywhere in the discipline...

It has been theorized that visual programming environments (VPE's) help to alleviate some of the early frustrations
associated with learning to program.  Many examples include Alice, Scratch, FIXME more...., however, very few have
been proposed that fall into the functional programming paradigm.  Why FP?
\end{comment}