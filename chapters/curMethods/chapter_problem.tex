\startchapter{Related Work}
\label{chapter:problem}

\newlength{\savedunitlength}
\setlength{\unitlength}{2em}


\setlength{\unitlength}{\savedunitlength}

\section{Misc Notes}

Description of introductory programming at University of Kiel can be found in \cite{Huch05}

Look at ACM/IEEE CS0/CS1/CS2 suggestions...  CC 2001 is at \cite{cc2001} 2008 interim revision is at \cite{cs2008}

Discussion of HtDP and HDMA(?)

Discussion of \cite{SICPbook} and it's influence is described in \cite{Flatt04}.  Famous critique of SICP is in \cite{Wadler87}

\cite{Bos98} mentions that Modula-3 is used as the first year language at the State University of New York at Stony Brook, and also has a rather lengthy discussion of why Java is a poor choice as first language.

\cite{Mahmoud04} has 


``The MIT computer science department in 1997 replaced C++ with Java as the primary software language that students were required to learn'' \cite{Benander04}  It now looks like they are moving to Python.

\cite{Benander04} has great numbers and figures on how various institutions have moved to Java.  Also provides very strong evidence to support the claim that Java is much easier to learn if you already know OO.  Also has support for the claim that Java is easy to learn if you are already experienced as a programmer, thus possibly lend more support to the notion of ``functional first'' (though be careful, it says Java is easy to learn if you know how to program, it doesn't say Java is easy to learn if you know FP).


Use of Python in CS1 and/or CS2:

\cite{Radenski06,Shannon03,Agarwal05,Agarwal08}.  MIT move from Scheme to Python: http://tech.mit.edu/V125/N65/coursevi.html

\cite{Necaise08} indicates that the popularity of python is the removal of syntax issues -- possible motivation for VPE's?