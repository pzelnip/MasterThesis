\startchapter{Related Work}
\label{chapter:problem}

\section{Misc Notes}

Description of introductory programming at University of Kiel can be found in \cite{Huch05}

Look at ACM/IEEE CS0/CS1/CS2 suggestions...  CC 2001 is at \cite{cc2001} 2008 interim revision is at \cite{cs2008}

Discussion of HtDP and HDMA(?)

Discussion of \cite{SICPbook} and it's influence is described in \cite{Flatt04}.  Famous critique of SICP is in \cite{Wadler87}

\cite{Bos98} mentions that Modula-3 is used as the first year language at the State University of New York at Stony Brook, and also has a rather lengthy discussion of why Java is a poor choice as first language.

\cite{Mahmoud04} has 


``The MIT computer science department in 1997 replaced C++ with Java as the primary software language that students were required to learn'' \cite{Benander04}  It now looks like they are moving to Python.

\cite{Benander04} has great numbers and figures on how various institutions have moved to Java.  Also provides very strong evidence to support the claim that Java is much easier to learn if you already know OO.  Also has support for the claim that Java is easy to learn if you are already experienced as a programmer, thus possibly lend more support to the notion of ``functional first'' (though be careful, it says Java is easy to learn if you know how to program, it doesn't say Java is easy to learn if you know FP).


Use of Python in CS1 and/or CS2:

\cite{Radenski06,Shannon03,Agarwal05,Agarwal08}.  MIT move from Scheme to Python: http://tech.mit.edu/V125/N65/coursevi.html

\cite{Necaise08} indicates that the popularity of python is the removal of syntax issues -- possible motivation for VPE's?

\cite{Pears07} has an excellent survey of various research questions in regards to the instruction of computer programming, and the design of the first year course.





Objects first?  See 3.4.3.2 of \cite{Pears07} for list of papers referring to Objects first approach

\section{Practices of Teaching Introductory Programming}

In this section we explore how computer programming has been, currently is, and will be taught at various post-secondary institutions.  A thorough survey of literature surrounding how programming has been and is currently being taught can be found in \cite{Pears07}.

\subsection{Curricula}

\subsection{Pedagogy}

Objects-first vs Imperative first vs functional first.  Contrast with design-first.  Difficulty of OO early.  FP early discussion.

Support that OO first is controversial is in \cite{Astrachan05}.  More heavy criticism of OO first is in \cite{Hu04}

There has been a move towards a ``design-first'' instructional style for the introductory course.  Traditionally introductory programming courses are taught by example.  The instructor introduces a new syntactic construct from the language being used, shows a number of examples using this construct, then exercises or assignments are given where students have to take the given code from the instructor and modify it to a new problem.  Some people feel that this approach emphasizes the language more than design, and given that the specific language in use is not the primary goal of the course, it has been argued that this creates courses whereby students walk away feeling as though programming is all about learning syntax, and not about designing creative solutions to interesting problems.  The consequence of this is that this approach makes the teaching of syntax explicit, and the teaching of design implicit, potentially causing courses to create student who can take existing code in Java or C++ (whichever language is used) and modify it to a new problem, but cannot design a solution to a problem from scratch.  \cite{Flatt04} outlines a course whereby design is made much more explicit to students, and has reported success in their approach.

There has also been research done which would indicate that perhaps beginning in the object-orientated paradigm is a less desirable choice than beginning in the functional paradigm\cite{Flatt04,Huch05}.  This has significant repercussions in terms of language choice, as (for example) Java is a heavily object-orientated language, and as such it has been argued is not the ideal choice for a first year course\cite{Bos98,Huch05}.  The TeachScheme/ReachJava project shows that ``functional first'' does not imply that object-orientation must be avoided in the first year of study\cite{Bloch08,teachScheme,Felleisen04}.


\subsection{Language Choice}

The choice of programming language is a critical one, though perhaps an overstated one.  Ideally the introductory programming course should be a course which teaches systematic thinking and problem solving, not ``how to write Java''.  Thus, while by necessity the introductory programming courses must make use of a programming language and environment, the primary goal of the introductory course is to teach programming in the abstract sense rather than specific syntax.

In the history of the field, computer science has seen a variety of languages employed at one time or another as being ``the teaching language of choice''.  For our purposes we shall outline three time-frames and list some of the languages most commonly in use during those times as well as summarize findings in regards to the successes and failures of those languages from a pedagogical standpoint.

\subsubsection{The Past}

FORTRAN, Scheme, Pascal

\subsubsection{Present Day}

C/C++, Java

\subsubsection{The Future}

As the result of the difficulties associated with the use of C/C++ and Java in the introductory programming course, many institutions are now switching from C/C++ and/or Java to other languages with the most common choice being the Python programming language\cite{python}.  Python has the advantage of being a scripted, interpreted language, thus students can ``try-out'' small expressions or snippets of code without having to get a complete source file free from errors.  It also has a relatively ``clean'' and simple syntax compared to the relatively verbose Java.  The use of whitespace for program structure encourages good indentation habits early as well.  With all these benefits it is not surprising to see some institutions beginning to adopt Python as the language for their introductory programming courses.  Perhaps the most notable adoption of Python happened at MIT, whose CS department had long been a strong proponent of the use of Scheme in introductory programming courses due to the adoption of \cite{SICPbook} as the introductory text.  Other examples of using Python in CS1/CS2 style courses and the possible benefits and drawbacks of doing so can be found in \cite{Radenski06,Shannon03,Agarwal05,Agarwal08}.

\section{Visual Programming Environments}

Alice, Scratch, Gamemaker, VFPE, Prograph

\section{Games and Computer Science Curricula}

