\startchapter{Related Work}
\label{chapter:problem}

\section{Misc Notes}

Description of introductory programming at University of Kiel can be found in \cite{Huch05}

Look at ACM/IEEE CS0/CS1/CS2 suggestions...  CC 2001 is at \cite{cc2001} 2008 interim revision is at \cite{cs2008}

Discussion of HtDP and HDMA(?)

Discussion of \cite{SICPbook} and it's influence is described in \cite{Flatt04}.  Famous critique of SICP is in \cite{Wadler87}

\cite{Bos98} mentions that Modula-3 is used as the first year language at the State University of New York at Stony Brook, and also has a rather lengthy discussion of why Java is a poor choice as first language.

\cite{Mahmoud04} has 


``The MIT computer science department in 1997 replaced C++ with Java as the primary software language that students were required to learn'' \cite{Benander04}  It now looks like they are moving to Python.

\cite{Benander04} has great numbers and figures on how various institutions have moved to Java.  Also provides very strong evidence to support the claim that Java is much easier to learn if you already know OO.  Also has support for the claim that Java is easy to learn if you are already experienced as a programmer, thus possibly lend more support to the notion of ``functional first'' (though be careful, it says Java is easy to learn if you know how to program, it doesn't say Java is easy to learn if you know FP).


MIT move from Scheme to Python: http://tech.mit.edu/V125/N65/coursevi.html

\cite{Necaise08} indicates that the popularity of python is the removal of syntax issues -- possible motivation for VPE's?

\cite{Pears07} has an excellent survey of various research questions in regards to the instruction of computer programming, and the design of the first year course.





Objects first?  See 3.4.3.2 of \cite{Pears07} for list of papers referring to Objects first approach

\section{Practices of Teaching Introductory Programming}

\begin{flushright}
\textit{So, if I look into my foggy crystal ball at the future of computing science education, I overwhelmingly see the depressing picture of ``Business as usual''.}
\\
Edsger W. Dijkstra \cite{Dijkstra89} \\
\end{flushright}


In this section we explore how computer programming has been, currently is, and will be taught at various post-secondary institutions.  A thorough survey of literature surrounding how programming has been and is currently being taught can be found in \cite{Pears07}.

\subsection{Curricula}

Introductory programming courses should always be considered within the context of the curriculum in which it lies.  Traditionally, the introductory programming course was one of the first (or was the first) course students undertaking a degree in Computer Science would enroll, as many future Computer Science courses assume basic programming experience as a basic skill.

In North America, many introductory computing courses follow the guidelines outlined in the Computing Curricula recommendations by the ACM and IEEE Computer Society Joint Task Force.  The latest full curriculum recommendation came in 2001 (known as CC2001), and can be found in \cite{cc2001}.  In 2008 an interim revision from the task force was submitted and is at \cite{cs2008}.  CC2001 identified 14 areas which comprised the body of knowledge for computer science at the undergraduate level.  These are listed in \tref{tab:CC2001BOK}.

\begin{table}
	\caption{The 14 Areas Which Comprise the CC2001 Body Of Knowledge For Computer Science}
\begin{tabular}{l}
Discrete Structures (DS) \\
Programming Fundamentals (PF) \\
Algorithms and Complexity (AL) \\
Architecture and Organization (AR) \\
Operating Systems (OS) \\
Net-Centric Computing (NC) \\
Programming Languages (PL) \\
Human-Computer Interaction (HC) \\
Graphics and Visual Computing (GV) \\
Intelligent Systems (IS) \\
Information Management (IM) \\
Social and Professional Issues (SP) \\
Software Engineering (SE) \\
Computational Science and Numerical Methods (CN) \\
\end{tabular}
	\label{tab:CC2001BOK}
\end{table}

Each of these areas are subdivided into units, and each unit consists of a number of topics.  Of the various units, some are given additional \emph{weight} in the recommendation by being designated as \emph{core} units, which are intended as being fundamental to \textbf{any} student of computer science irregardless of area of emphasis.  All units which are not designated as core units are referred to as \emph{elective} units.  While the core units comprise a fundamental set of topics for computer scientists, by themselves the core units (and the topics within) do not comprise a \emph{complete} set of material for a computer scientist -- they need to be supplemented with other elective units from other areas depending on the area of emphasis, the needs and goals of the education institution, etc.

Furthermore


two areas in particular are most heavily emphasized in the early courses of a computing degree -- Programming Fundamentals (PF) and Programming Languages (PL).  Programming fundamentals is one of only two\footnote{The other being Discrete Structures (DS)} areas outlined in CC2001 which are comprised entirely of core units.  These are listed in FIXME reference.


For many years, this task force has proposed a two-course introductory programming sequence.  These two courses are commonly generically referred to as CS1 and CS2.  Most educational institutions in North America have followed this pattern.
 Traditionally, the

\subsection{Pedagogy}

``While the curriculum defines what is to be taught, pedagogy deals with the manner in which teaching and learning are managed in order to facilitate desired learning outcomes''



Objects-first vs Imperative/Programming first vs functional first.  Contrast with design-first.  Difficulty of OO early.  FP early discussion.

Support that OO first is controversial is in \cite{Astrachan05}.  More heavy criticism of OO first is in \cite{Hu04} Excellent summary of pros and cons of OO first is in \cite{Lister06}

There has been a move towards a ``design-first'' instructional style for the introductory course.  Traditionally introductory programming courses are taught by example.  The instructor introduces a new syntactic construct from the language being used, shows a number of examples using this construct, then exercises or assignments are given where students have to take the given code from the instructor and modify it to a new problem.  Some people feel that this approach emphasizes the language more than design, and given that the specific language in use is not the primary goal of the course, it has been argued that this creates courses whereby students walk away feeling as though programming is all about learning syntax, and not about designing creative solutions to interesting problems.  The consequence of this approach is that it makes the teaching of syntax explicit and the teaching of design implicit, potentially causing courses to create students who can take existing code in Java or C++ (whichever language is used) and modify it to a new problem, but cannot design a solution to a problem from scratch.  \cite{Flatt04} outlines a course whereby design is made much more explicit to students, and has reported success in their approach.

There has also been research done which would indicate that perhaps beginning in the object-orientated paradigm is a less desirable choice than beginning in the functional paradigm\cite{Flatt04,Huch05}.  This has significant repercussions in terms of language choice, as (for example) Java is a heavily object-orientated language, and as such it has been argued is not the ideal choice for a first year course\cite{Bos98,Huch05}.  The TeachScheme/ReachJava project shows that ``functional first'' does not imply that object-orientation must be avoided in the first year of study \cite{Bloch08,teachScheme,Felleisen04}.


\subsection{Language Choice}

\begin{flushright}
\textit{The limits of my language mean the limits of my world.}
\\
Ludwig Wittgenstein \cite{Wittgenstein22} \\
\end{flushright}

The choice of programming language is a critical one, though perhaps an overstated one.  Ideally the introductory programming course should be a course which teaches systematic thinking and problem solving, not ``how to write Java''.  Thus, while by necessity the introductory programming courses must make use of a programming language and environment, the primary goal of the introductory course is to teach programming in the abstract sense rather than specific syntax.

In the history of the field, computer science has seen a variety of languages employed at one time or another as being ``the teaching language of choice''.  For our purposes we shall outline three time-frames and list some of the languages most commonly in use during those times as well as summarize findings in regards to the successes and failures of those languages from a pedagogical standpoint.

\subsubsection{The Past}

FORTRAN, Scheme, Pascal

\subsubsection{Present Day}

C/C++, Java

Problems with Java: \cite{Bos98,Benander04}

\subsubsection{The Future}

As the result of the difficulties associated with the use of C/C++ and Java in the introductory programming course, many institutions are now switching from C/C++ and/or Java to other languages.  The most common of these being the Python programming language \cite{python}.  Python has the advantage of being a scripted, interpreted language, thus students can ``try-out'' small expressions or snippets of code without having to get a complete source file free from errors.  It also has a relatively ``clean'' and simple syntax compared to the relatively verbose Java.  The use of whitespace for program structure encourages good indentation habits early as well.  With all these benefits it is not surprising to see some institutions beginning to adopt Python as the language for their introductory programming courses.  Perhaps the most notable adoption of Python happened at MIT, whose Computer Science department had long been a strong proponent of the use of Scheme in introductory programming courses due to the adoption of \cite{SICPbook} as the introductory text \cite{Thetech06}.  Other examples of using Python in CS1/CS2 style courses and the possible benefits and drawbacks of doing so can be found in \cite{Radenski06,Shannon03,Agarwal05,Agarwal08}.  While there have been successes with Python in the classroom, not even the strongest supporters of the language claim that it is the ``perfect choice''.

\section{Visual Programming Environments}

Alice, Scratch, Gamemaker, VFPE, Prograph

\cite{Hundhausen09} shows some support for the claim that a visual environment (or direct manipulation as he puts it) can provide positive ``transfer of training'' to a textual environment.  Draws a distinction between making programming easier to learn in a general sense and being able to transfer knowledge and understanding to a textual environment.

Use of Game Maker in educational setting at UVic can be found in \cite{Gooch08}.


\section{Games and Computer Science Curricula}
\label{chapter:problemSec:games}

In recent years there has been a notable decline in enrollment in Computer Science courses \cite{Manaris07,Vesgo07,Ward08,Bayliss09}.  As a result of this decline, many institutions are seeking out interesting and new ways to ``entice'' students to enroll in Computer Science courses and to rekindle interest in the discipline.  One such method that has been proposed is to incorporate computer games into Computer Science curricula.  \cite{Cliburn06} provides a summary of how games have been used in this fashion.

The rationale behind this move is that since video games are an exciting and compelling application of computer programming, that perhaps that interest in games can be leveraged by instructors in their introductory programming courses and beyond \cite{Overmars04,Sweedyk05,Barnes08}.  Furthermore, video games are a multi-billion dollar industry \cite{Wallace06}, and a common motivation for students in choosing a discipline are the employment opportunities a degree in the field would entail.  Since electronic entertainment is so prosperous, having curricula that targets students specifically for this field can be a compelling factor in one choosing a degree in Computer Science.

Some institutions have even taken this move a step further and are now offering gaming themes and concentrations in their degrees.  Recently, the University of Victoria added a ``Graphics and Gaming'' option to their Computer Science major program.  Other attempts at this are explored in \cite{Leutenegger07,Murray06,Zyda06}.  Most institutions that incorporate games into their courses or degrees have reported an increase in student enjoyment.  It has been shown that students prefer assignments based around games than ``traditional'' or ``story-telling'' assignments \cite{Cliburn08}.

%\cite{Cliburn08} provides strong evidence that in a general, aggregate study, students prefer assignments based around games than ``traditional'' or ``story-telling'' assignments.

While there have been successes reported surrounding the use of games, it is very much a controversial choice, as there are a number of concerns that have been raised in regards to gaming-centered curricula.  There is strong evidence to indicate that while gaming may spark initial interest in computing, it does not follow that this initial interest will translate into increases in students undertaking Computer Science majors.  A survey of 1,872 students conducted at a ``highly selective public technical university'' found that while 43\% of students indicated that games influenced their interest in computing, only 6.9\% realized that interest by becoming Computer Scientists \cite{DiSalvo09}.  Furthermore, while student interest generally seems to increase with assignments making use of games, few institutions have reported any noticible improvement in student performance.  Cliburn reported findings of a study he performed in the introductory programming course at Hanover College and found that students had a higher overall average score (95.1\% on average) on traditional assignments than on game-based assignments (89.1\% on average)\cite{Cliburn06}.  Interestingly, he also found that in spite of this, most students (78.9\%) still preferred game-based assignments over ``traditional'' assignments, perhaps a further testament to the motivating power of games as assignments.

Concerns have also been raised surrounding issues of gender and race.  One such concern suggests that games appeal more to male students than female, and that incorporation of games may alienate females from the discipline.  Specifically, there has been evidence to show that females tend to prefer games which are cooperative in nature rather than competitive, and that the latter can deter interest of women in games \cite{Camp02}.  This would indicate that educators must be careful about how they incorporate games into courses, and to design course materials with this concern in mind \cite{Carmichael08}.  Other works have adopted ``story-telling'' rather than games as being the vehicle of motivation, and have specifically explored this avenue with middle-school girls with great success \cite{Kelleher06}.

\begin{comment}
\cite{Natvig04,Barnes08} example of games being used
Some ideas about why creativity is important can be found in \cite{Farooq06}.
Games as a motivational tool: see ``International Conference on Game Development in Computer Science Education''.
An excellent outline of how games can be used successfully in CS curricula as well as how not only should we shoehorn games but also focus on fundamentals is summarized in \cite{Bayliss09}.  It also discusses what considerations should be made when incorporating games into courses.
\end{comment}
