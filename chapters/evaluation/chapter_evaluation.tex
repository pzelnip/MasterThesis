\startchapter{Evaluation of The Gem Cutter Environment}
\label{chapter:Exp}

\begin{flushright}
\textit{...a scientist must also be absolutely like a child.  If he sees a thing, he must say that he sees it, whether it was what he thought he was going to see or not.  See first, think later, then test.  But always see first.  Otherwise you will only see what you were expecting.}
\\
Douglas Adams \cite{Adams84} \\
\end{flushright}

\section{Applying the Cognitive Dimensions Framework to the Gem Cutter}

Apply cg framework to Gem Cutter....

\subsection{Abstraction Gradient}

As described in \sref{cgframeoutline} the Abstraction Gradient dimension tries to measure how much needs to be constructed
and learned in order to begin making the programming environment perform a task, and additionally how easy is it to add new abstractions to the environment.

\subsubsection{Discussion of Dimension}

In regards to the Gem Cutter, much like the VFPE, as a functional language there is a relatively small set of constructs to master before making the environment perform a task.  As a base minimum, one only needs to be familiar with the notion of a function, and how a gem is a visual representation of that concept, along with the mechanics of how to connect gems together on the tabletop.  As one wishes to do more sophisticated tasks, introduction of the additional gem types\footnote{Such as collector/emitter pairs, value gems, code gems, and record gems} will be required, however, this is still a relatively low minimum level of abstraction.  This is a great strength of the Gem Cutter, particularly from the perspective of a student learning to program, as little needs to be learned and mastered before beginning to make the environment perform basic tasks.

The Gem Cutter like most programming languages, would be an example of an abstraction-tolerant language, though only barely.  The only mechanism for introducing new abstractions is the ability to define new gems and use those gems in other gem designs.  Aside from this basic abstraction mechanic there is little or no support for introducing new abstractions.  In particular the lack of ability to introduce new types as discussed in \sref{sec:gemCutterShortcomings} greatly hinders ones ability to introduce new abstractions to reduce the complexity of larger problems.  This is mitigated somewhat by the introduction of code gems which allow one to introduce CAL code snippets at any point into a gem design.  Since CAL is very much an abstraction-tolerant language fully supporting the ability to introduce and define new types, this allows a ``loophole'' where one can bring all the expressive power of CAL to the Gem Cutter.  However, code gems will only be useful to those who already understand the CAL language.  This would be roughly analogous to requiring users of Alice to write Java code to create new object types to use in their Alice programs, which would seem rather cumbersome.

\subsubsection{Remedies, Workarounds, And Trade-offs}

Green suggests that a possible remedy for problems related to the abstraction gradient of an environment is to introduce incremental abstractions.  This would be where the environment has a low starting abstraction barrier (ie - start with a low minimum level of abstraction where little needs to be learned to begin working with the environment), and to allow the introduction of new abstractions that will aid users later.  To a certain degree the Gem Cutter does this already, as there is a relatively low minimum level of abstraction as mentioned above, and one can learn about the other gem types as they progress with the environment.  What is missing are enhanced facilities for newer abstractions, most notably in regards to types.  The introduction of a mechanism for defining a new type in a visual way would help to alleviate this aspect much more than the current ``workaround'' of using code gems.

%------------------------------------------------------------------------------------------------

\subsection{Closeness of Mapping}

As described in \sref{cgframeoutline} the dimension of Closeness of Mapping tries to measure the gap between the problem domain and the program domain. 

\subsubsection{Discussion of Dimension}

In terms of ``programming games'', the Gem Cutter imposes relatively few upon the user.  As a visual functional language, the notation of the Gem Cutter is rather concise.  Sometimes gem designs can become ``cluttered'' with an excessive number of visual entities, but this is more of an issue related to Diffuseness than of Closeness of Mapping.

Additionally, there are two other issues to consider from the perspective of Closeness of Mapping.  The first is the issue of the language's standard library support, and the second is what constructs the language allows to be built to improve the Closeness of Mapping.

With respect to the Gem Cutter, there is an extensive set of predefined gems provided to the user.  In particular, there is a great deal of library support in the Quark framework, CAL and the Gem Cutter for working with relational databases.  The DatabaseMetadata, Sql, SqlBuilder, and SqlParser modules are all imported by default into the Gem Cutter, and provide extensive support to close the gap between a problem in the domain of relational databases and the Gem Cutter environment itself.  In addition to these modules, the CAL language itself has a significant collection of modules for working in a variety of domains, any of which can be imported into the Gem Cutter.  The libarary support in the OpenQuark framework is not as robust as some more ``industry-proven'' languags (such as Sun's Java API), but is much more comprehensive and varied than the Prograph or VFPE environments.  

In terms of constructs to improve the Closeness of Mapping, much like the VFPE, the Gem Cutter can use higher-order functions to enable various programming ``idioms'' which can help greatly with changing the mapping from problem to program domain.

\subsubsection{Remedies, Workarounds, And Trade-offs}

There are no specific issues identified with the Gem Cutter with respect to Closness of Mapping.  Short of additional library support, there is little that can be introduced to the Gem Cutter to improve this dimension.

%------------------------------------------------------------------------------------------------

\subsection{Consistency}

As described in \sref{cgframeoutline} the dimension of Consistency tries to address how easy is it to infer the remaining parts of the language, once one has learned part of the language.

\subsubsection{Discussion of Dimension}

As noted in \sref{sec:prevAppCG}, visual languages tend to be very consistent due to the much simpler syntax.  The Gem Cutter is no different in this respect.  In particular, once one masters the metaphor of ``gem as function'' with inputs connecting to the left side of the gem, and the single output leaving the right side, the rest of the environment becomes very easy to learn as all gems follow this pattern.

Additionally, another issue related to consistency is library regularity.  Much like the Haskell language which inspired it, CAL (and as a result the Gem Cutter) keeps argument ordering very consistent.  For example, if a gem takes two arguments, one a function and the other a primitive type (such as an Integer), then the function argument will be the first (top-most) argument in the gem display.  This follows the common functional programming convention of having higher order functions come before primitive types in argument lists.  However, one very interesting thing to note however is that while this convention is followed, there is no restriction in the Gem Cutter in terms of in what order arguments are bound.  For example, consider the \code{map()} function common to virtually all functional programming languages.  The \code{map()} function is a function which takes two arguments, a function \code{f()}, and a list of items (\(x,x_0,x_1,...,x_n\)).  The result returned is the list with the function applied to each element or the list (\(f(x),f(x_0),f(x_1),...,f(x_n)\)).

In traditional textual functional languages such as Haskell or SML, one must bind the function argument before one binds the list argument.  The consequence of this is that we can use \code{map()} to create new functions of one argument -- a list.  We cannot however use \code{map()} to create a new function of a single list argument.  In the Gem Cutter, because we can bind arguments in any order, we do not have this limitation.  This would seem to be the best of both worlds, we have the flexibility and freedom to bind arguments in whichever order (thus allowing us greater expressivity), but we also have the consistency of argument ordering (though top down instead of left to right).

\subsubsection{Remedies, Workarounds, And Trade-offs}

The Gem Cutter is remarkably consistent, and as such there are no specific issues or ways that this dimension could be improved.

%------------------------------------------------------------------------------------------------

\subsection{Diffuseness}

As described in \sref{cgframeoutline} the dimension of Diffuseness (or terseness) tries to measure how many symbols are required to express a given meaning in the notation being examined.

\subsubsection{Discussion of Dimension}

With respect to the Gem Cutter, it is worth noting that the sample implementation by Green of the rocket trajectory program written in BASIC (seen in \pref{prog:basicRocket1}) was done in an iterative style, and since the Gem Cutter is a functional language there are no mechanisms for iteration instead requiring recursion to be used.  Thus while the implementation in Gem Cutter of the rocket trajectory program is based upon the BASIC version it is structured significantly differently.  Instead of having a single routine that ecompasses the entire problem, we created a gem called \code{rocket1()} which given the current state of the rocket\footnote{The vertical distance, vertical velocity, horizontal distance, horizontal velocity, and current mass} at time $t$, calculates the new state of the rocket at time $t + 1$ and returns this as a 6-tuple.  In some respects, this is a solution to the rocket trajectory problem by itself.  However, to more closely match the semantics of the BASIC version, a second gem was created called \code{rocketTester()} which generates the state of the rocket from time 0 to whatever time the rocket's vertical distance becomes negative.  Presumably, Kelso had to do something similar for the implementation in the VFPE of the rocket trajectory problem, however, the layout of his implementation was not supplied so we do not know how closely his approach matched that of the Gem Cutter implementation.

\begin{program}
\begin{verbatim}
Mass = 10000
Fuel = 50
Force = 400000
Gravity = 32
WHILE Vdist >= 0
      IF Tim = 11 THEN Angle = .3941
      IF Tim > 100 THEN Force = 0 ELSE Mass = Mass - Fuel
      Vaccel = Force*COS(Angle)/Mass � Gravity
      Vveloc = Vveloc + Vaccel
      Vdist = Vdist + Vveloc
      Haccel = Force*SIN(Angle)/Mass
      Hveloc = Hveloc + Haccel
      Hdist = Hdist + Hveloc
      PRINT Tim, Vdist, Hdist
      Tim = Tim + 1
WEND
STOP
\end{verbatim}
\caption{The Implementation of the Rocket Trajectory Problem in BASIC\cite{green96}}
\label{prog:basicRocket1}
\end{program}

The breakdown of graphic entities in the two gems can be seen in \tref{tab:GCrocket1}.  The complete solution of both gems for the rocket trajectory problem required 161 different graphic entities, far more than the 131 for Prograph, 104 for Labview, and 98 for the VFPE.  Note that connector lines were included in the totals for the Gem Cutter implementation.  This was different than the totals for the VFPE where connectors were not included in the total due to the fact that connection lines were not established by the user.  Connection lines were included in the total for Prograph, however, this was due to the fact that when one wishes to move a component in a Prograph layout, they also need to manually adjust the connections.  In Gem Cutter, while the connections between gems need to be established by the user, once they are established movement of gems around the tabletop does will cause the connections to be redrawn automatically.  If we leave out the connections, then the total for the Gem Cutter implementation falls to 88 graphic entities, easily the most terse of the four considered.

\begin{table}
	\caption{The Breakdown of Graphic Entities in the Gem Cutter Solution to the Rocket Trajectory Problem}
\begin{tabular}{|l|m{2.5cm}|m{3.5cm}||m{3cm}|}
\hline
\textbf{Gem Type} & \textbf{\code{rocket1()} Count} & \textbf{\code{rocketTester()} Count} & \textbf{Total For Both} \\
\hline
Connectors & 59 & 14 & 73 \\
Emitter Gems & 27 & 1 & 28 \\
Function Gems & 19 & 5 & 24 \\
Value Gems & 9 & 7 & 16 \\
Collector Gems & 12 & 1 & 13 \\
Record Selection Gems & 6 & 1 & 7 \\
\hline \hline
\textbf{Totals} & 132 & 29 & \underline{161} \\
\hline
\end{tabular}
	\label{tab:GCrocket1}
\end{table}

Also note that the single target gem was not counted in these numbers as it is a graphic entity that always appears in any gem design (it is not added by the user).  Given that there is only a single target gem for any gem design, the inclusion or exclusion of this entity will not sigificantly alter the graphic count either way.

The use of record selection gems to parse a tuple into subsequent parts incurs a significant number of entities as well.  While the count in \tref{tab:GCrocket1} does not seem to indicate this (as there are only 7 record selection gems), what is missing from this total is the fact that connected to the record selection gems are two connectors, and a collector and emitter pair for each (an emitter for the tuple being parsed, and a collector to ``name'' the parsed value).  Thus breaking the 6-tuple rocket state into six separate values in \code{rocket1()} adds a total of 30 entities (including connectors, 18 if we omit connectors).

In terms of screen space, the implementation of the \code{rocket1()} gem would not fit on a single 1280x960 resolution screen, instead scrolling roughly halfway past the end of the screen.  The \code{rocketTester()} gem however fit easily on a single screen, and can be seen in \fref{gcRocketTester1}.

\insertFigure{4}{gcRocketTester1}{The \code{rocketTester()} Gem}

\subsubsection{Remedies, Workarounds, And Trade-offs}



%------------------------------------------------------------------------------------------------

\subsection{Error-Proneness}

As described in \sref{cgframeoutline} the dimension of  ....

\subsubsection{Discussion of Dimension}

\subsubsection{Remedies, Workarounds, And Trade-offs}

%------------------------------------------------------------------------------------------------

\subsection{Hard Mental Operations}

As described in \sref{cgframeoutline} the dimension of  ....

\subsubsection{Discussion of Dimension}

\subsubsection{Remedies, Workarounds, And Trade-offs}

%------------------------------------------------------------------------------------------------

\subsection{Hidden Dependencies}

As described in \sref{cgframeoutline} the dimension of Hidden Dependencies tries to answer the question ``Is every dependency overtly indicated in both directions? Is the indication perceptual or only symbolic?''.

\subsubsection{Discussion of Dimension}

In regards to the Gem Cutter, there are a few points where the software very much aids this dimension.  One example is the use of diamond and line connections between gems.  This visual representation of the dependency between two gems \emph{within a single design} is made explicit via the notation.  Put another way, the added visual cues to the notation which makes the dependency explicit improves this dimension of the Gem Cutter.  This is very similar to the findings by Green in regards to LabView and (on a local level) Prograph as discussed in \sref{sec:prevAppCG}, where the ``linking wires'' of LabView served a similar purpose.

The use of visual cues to indicate types also aids this dimension.  For example, there is a dependency between collector gems (which collect a value) and emitter gems (which emit the value collected).  The fact that both collector and emitter gems are annotated and labeled provide a visual cue that there is a dependency between the collector and emitter.

A problem with regards to hidden dependencies within Gem Cutter is the issue of type dependencies between separate gem designs discussed in \sref{sec:gemCutterShortcomings}.  There is no mechanism within Gem Cutter to see what gems a particular gem is dependent upon short of examining the gem design.  Even if there were, the fact that the dependency relationship is transitive would require such a facility to be able to discover dependency relationships multiple levels deep.  For example, if \code{a()} calls \code{b()} which calls \code{c()}, it is not enough to know that \code{c()} is dependent upon \code{b()}, but \code{a()} as well.  There is the facility for determining what gems depend upon a given gem (that is, ``what gems depend upon this gem?''), but not the other way around (that is, ``what gems does this gem depend upon?'').  To use the language of the cognitive dimensions framework: not every dependency is overtly indicated in \emph{both} directions.

\subsubsection{Remedies, Workarounds, And Trade-offs}

Green suggests three possible remedies to the Hidden Dependencies dimension: 
\begin{inparaenum}[(a)]
	\item adding cues to the notation;
	\item highlighting different information; and 
	\item providing extra tools.
\end{inparaenum}

In regards to the type dependency issue it is difficult to envision what possible additional cues could be added to the interface without ``cluttering'' the interface.  The possibility of an extra tool which allows for one to see a tree-like structure of all gems which a given gem depends upon could help.

%------------------------------------------------------------------------------------------------

\subsection{Premature Commitment}

As described in \sref{cgframeoutline} the dimension of  ....

\subsubsection{Discussion of Dimension}

\subsubsection{Remedies, Workarounds, And Trade-offs}

%------------------------------------------------------------------------------------------------

\subsection{Progressive Evaluation}

As described in \sref{cgframeoutline} the dimension of  ....

\subsubsection{Discussion of Dimension}

\subsubsection{Remedies, Workarounds, And Trade-offs}

%------------------------------------------------------------------------------------------------

\subsection{Role-Expressiveness}

As described in \sref{cgframeoutline} the dimension of  ....

\subsubsection{Discussion of Dimension}

\subsubsection{Remedies, Workarounds, And Trade-offs}

%------------------------------------------------------------------------------------------------

\subsection{Secondary Notation}

As described in \sref{cgframeoutline} the dimension of  ....

\subsubsection{Discussion of Dimension}

\subsubsection{Remedies, Workarounds, And Trade-offs}

%------------------------------------------------------------------------------------------------

\subsection{Viscosity}

As described in \sref{cgframeoutline} the dimension of  ....

\subsubsection{Discussion of Dimension}

\subsubsection{Remedies, Workarounds, And Trade-offs}

%------------------------------------------------------------------------------------------------

\subsection{Visibility}

As described in \sref{cgframeoutline} the dimension of  ....

\subsubsection{Discussion of Dimension}

\subsubsection{Remedies, Workarounds, And Trade-offs}


\subsection{notes}

languages which do not require variables to be declared before they are used.  For example in fixme reference, we three implementations of the well known quadratic equation, one in Java, one in Python, and the last in Gem Cutter.  The two textual versions contain the same error in both: the identifier ``disc'' is misspelled on its second use, and thus does not refer to the value desired (the discriminant).  However, in SML, because identifiers must be declared via a val-binding before they are used this error is caught at compile time, whereas in Python (which does not require variables be declared before use), this results in a runtime error the programmer must debug.  Thus one could argue that SML is less error-prone than Python.  However, the Gem Cutter version completely avoids the possibility of the error, as one cannot add an emitter gem unless there is already a corresponding collector gem on the tabletop.


Role expressiveness:
It has been shown that recognition of beacons in code is strongly correlated to programmer experience.  A study done by Wiedenbeck found that while 79\% of experienced programmers recalled significantly more of the beacon lines in a program, only 14\% of novices could do the same \cite{Wiedenbeck91}.  As the focus of this thesis is on the teaching of introductory programming concepts to students new to computer programming, ``beacons'' will be of less importance in our analysis.


Hidden dependencies -- gem A depends on gem B, so changes to B can break A.
Hidden dependencies -- diamond and line makes connections between gems much more visually apparent.
Premature Commitment -- problem of not being able to save an incomplete gem (commitment to order of creation)
Progressive Evaluation -- ability to evaluate any subexpression at any time by right-clicking and choosing ``run gem''

viscosity - can't open a new gem design until the current is finished

visibility - can't juxtapose two gem designs simultaneously.

viscosity - layout problems addressed by automatically rearranging items


diffuseness - record selection adds many graphical entities, if/else blocking with multiple assignment statements in each block

secondary notation - limited commenting ability. Can describe a gem as a whole, but not subparts of the gem (for example annotating constant value gems would be nice, this can be done by using collector/emitter pairs to name them, but this negatively impacts diffuseness)

\begin{comment}

\section{Outline Of An Experiment For Evaluating The Effectiveness Of The Gem Cutter}

In the preceding section we explored a qualitative evaluation of the Gem Cutter with a bias toward a pedagogical viewpoint.  However, a useful compliment to this would be to explore a more \emph{quantitative} evaluation of the environment.  In this section we will outline an experiment which could provide a meaningful empirical evaluation of the Gem Cutter.

\subsection{Avenues of Evaluation}

\cite{Kelso02} outlined an experiment for his VFPE in which he outlined two possible avenues of for empirical evaluation of the VFPE:

\begin{enumerate}
	\item Evaluating the general adequacy of the environment
	\item Investigating Visual Programming proper
\end{enumerate}

The first is focused on conducting experiments to evaluate the general adequacy of the programming environment.  Such experiments would focus on showing quantitatively that for programmers experienced with textual languages, that the visual environment is not significantly worse than the textual alternative.  That is, that the environment is ``feature-equivalent'' to a given textual environment.

The second is for the purpose of exploring the textual/visual division.  That is, experiments designed here would be for the purpose of identifying differences that are due solely to the respective modes of display and the programming environment tools.  This was the type of experiment that Kelso outlined for his VFPE.  This requires a textual environment which is roughly equivalent to the VFPE.  
Given that the VFPE from a semantics and abstract syntax viewpoint is very similar to Haskell, Kelso believed that an experiment comparing the two would reveal differences between the visual and textual forms of representation.  Given that the Gem Cutter truly is the visual representation of CAL code, it would seem that an experiment of this form with the Gem Cutter and CAL would be more telling of the differences between the visual and the textual.

In addition to these, we can foresee other avenues for empirical evaluation of the Gem Cutter, most notably that of functional versus object-orientated programming.  There have already been studies which have explored the differences between the two paradigms (FIXME - references??), but they have largely been based solely in the textual world.  Similar experiments using the Gem Cutter (a visual environment rooted in the functional paradigm) and one of the common object-orientated environments (such as Prograph, Alice, or Scratch) would be an interesting addition to the work exploring the difference between the two paradigms.

For our purposes as educators however, we are more interested about how effective the Gem Cutter can be as a learning tool for students.  As such, we shall identify an experiment which explores two other avenues of exploration.  The first is to assess if the Gem Cutter can meet a typical set of learning objectives (ie - ``Does the student learn the concept?''), and the second assesses the transfer of skill to another environment (ie - ``Can the student apply what he/she learned to a different language?'').

\subsection{Difficulties}

Assessing which of two competing methods of instruction is more effective seems at first glance to be a simple task.  Simply teach one subject with both methods and compare the results of the two approaches.  However, as outlined by McKeachie in \cite{teachingTips} there are some difficulties or ``hidden traps'' that arise in trying to evaluate two different approaches to instruction.

The first category of problems are methodological, 

From Teaching Tips, \cite{teachingTips}:

methodological problems:

\begin{itemize}
	\item Hawthorne effect - instructor enthusiasm can influence quality of instruction, students react differently when they know they are being taught by an experimental method -- necessitates evaluation over period of time
	\item Finding control group -- different classes have different makeups of students, some instructors are better than others (so if method is successful how much is the method and how much the instructor)
\end{itemize}

criterion problem - you want to ``know what each group learned that the other did not.  Thus a comparison of the lecture method with a discussion method based on a common final examination from a textbook does not really compare what the two groups of students learned in their different classes, but rather what they learned from reading the text.''  ''criterion measure should sample progress on \textit{all} goals, not just a small sample chosen for a particular method.''  Student motivation -- students are motivated to get good grades, and thus may compensate for poor instruction, thereby obfuscating research results.

\subsection{Preparation}


\subsection{Group Selection}


\subsection{Measurement}

\section{Misc Notes}

\cite{Bayliss09} in 3.7 talks about important points in assessing the use of games, which may be relevant to this experiment.

\end{comment}