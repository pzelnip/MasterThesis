\startchapter{Evaluation of The Gem Cutter Environment}
\label{chapter:Exp}

\begin{flushright}
\textit{...a scientist must also be absolutely like a child.  If he sees a thing, he must say that he sees it, whether it was what he thought he was going to see or not.  See first, think later, then test.  But always see first.  Otherwise you will only see what you were expecting.}
\\
Douglas Adams \cite{Adams84} \\
\end{flushright}

\section{Applying the Cognitive Dimensions Framework to the Gem Cutter}

Apply cg framework to Gem Cutter....

\subsection{Abstraction Gradient}

\subsubsection{Discussion of Dimension}

As described in \sref{cgframeoutline} the Abstraction Gradient dimension tries to measure how much needs to be constructed
and learned in order to begin making the programming environment perform a task, and additionally how easy is it to add new abstractions to the environment.

In regards to the Gem Cutter, much like the VFPE, as a functional language there is a relatively small set of constructs to master before making the environment perform a task.  As a base minimum, one only needs to be familiar with the notion of a function, and how a gem is a visual representation of that concept, along with the mechanics of how to connect gems together on the tabletop.  As one wishes to do more sophisticated tasks, introduction of the additional gem types\footnote{Such as collector/emitter pairs, value gems, code gems, and record gems} will be required, however, this is still a relatively low minimum level of abstraction.  This is a great strength of the Gem Cutter, particularly from the perspective of a student learning to program, as little needs to be learned and mastered before beginning to make the environment perform basic tasks.

The Gem Cutter like most programming languages, would be an example of an abstraction-tolerant language, though only barely.  The only mechanism for introducing new abstractions is the ability to define new gems and use those gems in other gem designs.  Aside from this basic abstraction mechanic there is little or no support for introducing new abstractions.  In particular the lack of ability to introduce new types as discussed in \sref{sec:gemCutterShortcomings} greatly hinders ones ability to introduce new abstractions to reduce the complexity of larger problems.  This is mitigated somewhat by the introduction of code gems which allow one to introduce CAL code snippets at any point into a gem design.  Since CAL is very much an abstraction-tolerant language fully supporting the ability to introduce and define new types, this allows a ``loophole'' where one can bring all the expressive power of CAL to the Gem Cutter.  However, code gems will only be useful to those who already understand the CAL language.  This would be roughly analogous to requiring users of Alice to write Java code to create new object types to use in their Alice programs, which would seem rather cumbersome.

\subsubsection{Remedies, Workarounds, And Trade-offs}

Green suggests that a possible remedy for problems related to the abstraction gradient of an environment is to introduce incremental abstractions.  This would be where the environment has a low starting abstraction barrier (ie - start with a low minimum level of abstraction where little needs to be learned to begin working with the environment), and to allow the introduction of new abstractions that will aid users later.  To a certain degree the Gem Cutter does this already, as there is a relatively low minimum level of abstraction as mentioned above, and one can learn about the other gem types as they progress with the environment.  What is missing are enhanced facilities for newer abstractions, most notably in regards to types.  The introduction of a mechanism for defining a new type in a visual way would help to alleviate this aspect much more than the current ``workaround'' of using code gems.

\subsection{Hidden Dependencies}

\subsubsection{Discussion of Dimension}

As described in \sref{cgframeoutline} the dimension of Hidden Dependencies tries to answer the question ``Is every dependency overtly indicated in both directions? Is the indication perceptual or only symbolic?''.

In regards to the Gem Cutter, there are a few points where the software very much aids this dimension.  One example is the use of diamond and line connections between gems.  This visual representation of the dependency between two gems \emph{within a single design} is made explicit via the notation.  Put another way, the added visual cues to the notation which makes the dependency explicit improves this dimension of the Gem Cutter.  This is very similar to the findings by Green in regards to LabView and (on a local level) Prograph as discussed in \sref{sec:prevAppCG}, where the ``linking wires'' of LabView served a similar purpose.

The use of visual cues to indicate types also aids this dimension.  For example, there is a dependency between collector gems (which collect a value) and emitter gems (which emit the value collected).  The fact that both collector and emitter gems are annotated and labeled provide a visual cue that there is a dependency between the collector and emitter.

A problem with regards to hidden dependencies within Gem Cutter is the issue of type dependencies between separate gem designs discussed in \sref{sec:gemCutterShortcomings}.  There is no mechanism within Gem Cutter to see what gems a particular gem is dependent upon short of examining the gem design.  Even if there were, the fact that the dependency relationship is transitive would require such a facility to be able to discover dependency relationships multiple levels deep.  For example, if \code{a()} calls \code{b()} which calls \code{c()}, it is not enough to know that \code{c()} is dependent upon \code{b()}, but \code{a()} as well.  There is the facility for determining what gems depend upon a given gem (that is, ``what gems depend upon this gem?''), but not the other way around (that is, ``what gems does this gem depend upon?'').  To use the language of the cognitive dimensions framework: not every dependency is overtly indicated in \emph{both} directions.

\subsubsection{Remedies, Workarounds, And Trade-offs}

Green suggests three possible remedies to the Hidden Dependencies dimension: 
\begin{inparaenum}[(a)]
	\item adding cues to the notation;
	\item highlighting different information; and 
	\item providing extra tools.
\end{inparaenum}

In regards to the type dependency issue it is difficult to envision what possible additional cues could be added to the interface without ``cluttering'' the interface.  The possibility of an extra tool which allows for one to see a tree-like structure of all gems which a given gem depends upon could help.


\subsection{notes}

languages which do not require variables to be declared before they are used.  For example in fixme reference, we three implementations of the well known quadratic equation, one in Java, one in Python, and the last in Gem Cutter.  The two textual versions contain the same error in both: the identifier ``disc'' is misspelled on its second use, and thus does not refer to the value desired (the discriminant).  However, in SML, because identifiers must be declared via a val-binding before they are used this error is caught at compile time, whereas in Python (which does not require variables be declared before use), this results in a runtime error the programmer must debug.  Thus one could argue that SML is less error-prone than Python.  However, the Gem Cutter version completely avoids the possibility of the error, as one cannot add an emitter gem unless there is already a corresponding collector gem on the tabletop.


Role expressiveness:
It has been shown that recognition of beacons in code is strongly correlated to programmer experience.  A study done by Wiedenbeck found that while 79\% of experienced programmers recalled significantly more of the beacon lines in a program, only 14\% of novices could do the same \cite{Wiedenbeck91}.  As the focus of this thesis is on the teaching of introductory programming concepts to students new to computer programming, ``beacons'' will be of less importance in our analysis.


Hidden dependencies -- gem A depends on gem B, so changes to B can break A.
Hidden dependencies -- diamond and line makes connections between gems much more visually apparent.
Premature Commitment -- problem of not being able to save an incomplete gem (commitment to order of creation)
Progressive Evaluation -- ability to evaluate any subexpression at any time by right-clicking and choosing ``run gem''

viscosity - can't open a new gem design until the current is finished

visibility - can't juxtapose two gem designs simultaneously.

\section{Outline Of An Experiment For Evaluating The Effectiveness Of The Gem Cutter}

In the preceding section we explored a qualitative evaluation of the Gem Cutter with a bias toward a pedagogical viewpoint.  However, a useful compliment to this would be to explore a more \emph{quantitative} evaluation of the environment.  In this section we will outline an experiment which could provide a meaningful empirical evaluation of the Gem Cutter.

\subsection{Avenues of Evaluation}

\cite{Kelso02} outlined an experiment for his VFPE in which he outlined two possible avenues of for empirical evaluation of the VFPE:

\begin{enumerate}
	\item Evaluating the general adequacy of the environment
	\item Investigating Visual Programming proper
\end{enumerate}

The first is focused on conducting experiments to evaluate the general adequacy of the programming environment.  Such experiments would focus on showing quantitatively that for programmers experienced with textual languages, that the visual environment is not significantly worse than the textual alternative.  That is, that the environment is ``feature-equivalent'' to a given textual environment.

The second is for the purpose of exploring the textual/visual division.  That is, experiments designed here would be for the purpose of identifying differences that are due solely to the respective modes of display and the programming environment tools.  This was the type of experiment that Kelso outlined for his VFPE.  This requires a textual environment which is roughly equivalent to the VFPE.  
Given that the VFPE from a semantics and abstract syntax viewpoint is very similar to Haskell, Kelso believed that an experiment comparing the two would reveal differences between the visual and textual forms of representation.  Given that the Gem Cutter truly is the visual representation of CAL code, it would seem that an experiment of this form with the Gem Cutter and CAL would be more telling of the differences between the visual and the textual.

In addition to these, we can foresee other avenues for empirical evaluation of the Gem Cutter, most notably that of functional versus object-orientated programming.  There have already been studies which have explored the differences between the two paradigms (FIXME - references??), but they have largely been based solely in the textual world.  Similar experiments using the Gem Cutter (a visual environment rooted in the functional paradigm) and one of the common object-orientated environments (such as Prograph, Alice, or Scratch) would be an interesting addition to the work exploring the difference between the two paradigms.

For our purposes as educators however, we are more interested about how effective the Gem Cutter can be as a learning tool for students.  As such, we shall identify an experiment which explores two other avenues of exploration.  The first is to assess if the Gem Cutter can meet a typical set of learning objectives (ie - ``Does the student learn the concept?''), and the second assesses the transfer of skill to another environment (ie - ``Can the student apply what he/she learned to a different language?'').

\subsection{Difficulties}

Assessing which of two competing methods of instruction is more effective seems at first glance to be a simple task.  Simply teach one subject with both methods and compare the results of the two approaches.  However, as outlined by McKeachie in \cite{teachingTips} there are some difficulties or ``hidden traps'' that arise in trying to evaluate two different approaches to instruction.

The first category of problems are methodological, 

From Teaching Tips, \cite{teachingTips}:

methodological problems:

\begin{itemize}
	\item Hawthorne effect - instructor enthusiasm can influence quality of instruction, students react differently when they know they are being taught by an experimental method -- necessitates evaluation over period of time
	\item Finding control group -- different classes have different makeups of students, some instructors are better than others (so if method is successful how much is the method and how much the instructor)
\end{itemize}

criterion problem - you want to ``know what each group learned that the other did not.  Thus a comparison of the lecture method with a discussion method based on a common final examination from a textbook does not really compare what the two groups of students learned in their different classes, but rather what they learned from reading the text.''  ''criterion measure should sample progress on \textit{all} goals, not just a small sample chosen for a particular method.''  Student motivation -- students are motivated to get good grades, and thus may compensate for poor instruction, thereby obfuscating research results.

\subsection{Preparation}


\subsection{Group Selection}


\subsection{Measurement}

\section{Misc Notes}

\cite{Bayliss09} in 3.7 talks about important points in assessing the use of games, which may be relevant to this experiment.