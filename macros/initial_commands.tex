% These commands are used
% immediate after \begin{document}

% Setup old style numerals macro
\newcommand\osn[1]{\oldstylenums{#1}\xspace}

% Setup a definition style macro
\newcommand\defn[1]{\textbf{#1}}

\newcommand{\gt}{\(>\)}		% less than sign shorthand
\newcommand{\lt}{\(<\)}		% greater than sign shorthand
\newcommand{\vbar}{\(|\)}	% vertical bar shorthand
\newcommand{\code}[1]{{\tt #1}}
\newcommand{\ie}{i.e.\ }

\newcommand{\fref}[1]{Figure~\ref{#1}}
\newcommand{\pref}[1]{Program~\ref{#1}}
\newcommand{\tref}[1]{Table~\ref{#1}}
\newcommand{\sref}[1]{Section~\ref{#1}}
\newcommand{\aref}[1]{Appendix~\ref{#1}}
\newcommand{\filepath}[1]{{\tt #1}}

\newcommand{\insertFigureLoc}[4]{
\begin{figure}[#1]
\begin{center}
\includegraphics[width=#2in]{Figures/#3}
\end{center}
\caption{#4}
\label{#3}
\end{figure}
}	

\newcommand{\insertFigure}[3]{
	\insertFigureLoc{ht}{#1}{#2}{#3}
}


\newcommand{\labExer}[6]{
	\section{#1}
	
	\subsection*{Learning Objectives}

	By the end of this lesson, participants will be able to:

	#2

	\subsection*{Task Type (Self-Directed or Guided)}

	#3

	\subsection*{Pre-Requisite Knowledge/Skills Needed}

	Participants must already know/be able to:

	#4
	
	\subsection*{Extra Environment Assumptions}

	#5

	\subsection*{Outline of Tasks}

	#6
}

\newcommand{\labExerWithStudent}[7]{
	\labExer{#1}{#2}{#3}{#4}{#5}{#6}
	
	#7
}


% the following 3 lines are for the "program" float type which is used for source
% code listings.  It requires the ``float'' package.
\floatstyle{ruled}
\newfloat{program}{thp}{lop}
\floatname{program}{Program}
